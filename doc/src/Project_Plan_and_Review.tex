\documentclass[12pt, a4paper]{article}

% Page layout
\usepackage[left=25mm, right=10mm, top=20mm, bottom=20mm]{geometry}

% Fonts and Language
\usepackage{fontspec}
% According to HSE standards, Times New Roman is required
% If not available, use Liberation Serif (free alternative) or DejaVu Serif
\IfFontExistsTF{Times New Roman}{%
  \setmainfont{Times New Roman}
}{%
  \IfFontExistsTF{Liberation Serif}{%
    \setmainfont{Liberation Serif}
  }{%
    \setmainfont{DejaVu Serif}
  }%
}
\usepackage[english, russian]{babel}

% Spacing and Indentation
\usepackage{setspace}
\onehalfspacing
\usepackage{indentfirst}
\setlength{\parindent}{1.25cm}

% Lists
\usepackage{enumitem}
\setlist{nolistsep} % Reduce space between list items

% Headings
\usepackage{titlesec}
\titleformat{\section}
  {\normalfont\fontsize{14}{14}\bfseries}{\thesection}{1em}{}
\titleformat{\subsection}
  {\normalfont\fontsize{12}{12}\bfseries}{\thesubsection}{1em}{}
\titleformat{\subsubsection}
  {\normalfont\fontsize{12}{12}\bfseries}{\thesubsubsection}{1em}{}

% Graphics
\usepackage{graphicx}

% Table of contents formatting
\usepackage{tocloft}
\renewcommand{\cftsecleader}{\cftdotfill{\cftdotsep}}
\setcounter{tocdepth}{3}  % Show sections, subsections, and subsubsections

% Bibliography - ГОСТ 7.0.5-2008
\usepackage[numbers, sort&compress]{natbib}

\begin{document}

% Table of contents
\tableofcontents
\newpage

\section{Календарный план работы над ВКР}

\noindent \textbf{Тема:} Генерация персонализированных учебных конспектов по предметам ЕГЭ на основе языковых моделей и пользовательской аналитики

\noindent \textbf{Студенты:} Ахундов Алексей Назимович, Жарковский Дмитрий Андреевич

\noindent \textbf{Руководитель:} Ахтямов Павел Ибрагимович

\noindent \textbf{Срок выполнения:} 2025–2026 учебный год 


\subsection{Этап 1: Подготовительный и проектирование (Ноябрь — Декабрь 2025)}

\noindent \textbf{Цель:} Утверждение концепции и техническое проектирование системы.

\noindent \textbf{Задачи:}
\begin{itemize}[leftmargin=*]
    \item Согласование детального плана работы и темы с научным руководителем (до 01.12.2025).
    \item Сбор и анализ научной и технической литературы по теме (LLM в образовании, MCP, персонализация обучения).
    \item Изучение спецификации Model Context Protocol (MCP) от Anthropic.
    \item Проектирование архитектуры (схема LLM - MCP Server - Database, модель данных профиля ученика).
    \item Подготовка тестовых данных: эталонные конспекты и обезличенные профили учеников.
\end{itemize}

\noindent \textbf{Результат этапа:} Утвержденный план, черновик Введения/Обзора, схема БД и API.

\subsection{Этап 2: Разработка MVP и MCP-сервера (Январь — Февраль 2026)}

\noindent \textbf{Цель:} Создание работающего прототипа.

\noindent \textbf{Задачи:}
\begin{itemize}[leftmargin=*]
    \item Реализация MCP-сервера на Python (Resources: доступ к профилю, Tools: запрос статистики).
    \item Разработка логики генерации и системных промптов с использованием контекста MCP.
    \item Создание веб-сервиса (интерфейс для генерации и просмотра конспектов).
    \item Первичное тестирование.
\end{itemize}

\noindent \textbf{Результат этапа:} Рабочий прототип (MVP), генерирующий уникальные конспекты под профиль.

\subsection{Этап 3: Эксперименты и доработка (Март — Апрель 2026)}

\noindent \textbf{Цель:} Оценка качества и написание текста ВКР.

\noindent \textbf{Задачи:}
\begin{itemize}[leftmargin=*]
    \item Сравнительный эксперимент (Группы: А - Методист, Б - Система, В - Чистая LLM).
    \item Сбор метрик (фактологическая точность, персонализация, читаемость).
    \item Оптимизация системы по итогам тестов.
    \item Написание глав ВКР: Архитектура, Реализация, Эксперименты.
\end{itemize}

\noindent \textbf{Результат этапа:} Готовый продукт, графики/таблицы экспериментов, черновик ВКР.

\subsection{Этап 4: Оформление и защита (Май — Июнь 2026)}

\noindent \textbf{Цель:} Финализация и защита.

\noindent \textbf{Задачи:}
\begin{itemize}[leftmargin=*]
    \item Сборка полного текста ВКР, нормоконтроль, антиплагиат.
    \item Подготовка презентации и демо-стенда.
    \item Предзащита и финальные правки.
    \item Защита ВКР.
\end{itemize}

\noindent \textbf{Результат этапа:} Защищенная ВКР.

\newpage

\section{Обзор литературы}

\noindent \textbf{Тема:} Генерация персонализированных учебных конспектов по предметам ЕГЭ на основе языковых моделей и пользовательской аналитики

\subsection{Краткий обзор литературы и существующих решений}

\noindent Несмотря на бум генеративного ИИ в образовании, интеграция технологий в учебный процесс только начинается. Рынок делят гиганты: Khanmigo \cite{khanmigo}, Q-Chat \cite{qchat} и клоны GPT-4. Как отмечают Hu et al. (2025), тренд смещается от простых ответов к «Сократическим площадкам», где ИИ направляет ученика \cite{hu2025}.

Однако существующие системы имеют критичные для подготовки к ЕГЭ недостатки:

\begin{enumerate}[label=\arabic*.]
    \item \textbf{Формат чат-бота.} Диалог «вопрос-ответ» хорош для точечных проблем, но не подходит для системного изучения тем. Чат-боты не дают целостных, структурированных материалов (лонгридов).
    \item \textbf{Нет «памяти» (Context Awareness).} LLM обычно не имеют доступа к истории успеваемости за год. Адаптивность часто сводится к подбору задач, а не теории \cite{li2024}. Модель не знает о хронических ошибках ученика, если ей об этом не сказать.
    \item \textbf{Отсутствие стандартов.} Универсальные модели без жестких ограничений могут генерировать верные, но избыточные или не соответствующие кодификатору ЕГЭ объяснения, путая школьника.
\end{enumerate}

\noindent \textbf{Вывод:} Нужен инструмент, генерирующий не реплики в чате, а полноценные методички (конспекты), жестко привязанные к ФИПИ и адаптированные под историю ошибок конкретного ученика.

\subsection{Ключевые ожидаемые результаты (Новизна и Уникальность)}

\noindent Разрабатываемая система решает эти проблемы и обладает следующей новизной:

\subsubsection{Технологическая новизна: Model Context Protocol (MCP)}
\noindent Вместо громоздких RAG-схем используется MCP от Anthropic \cite{anthropic2025}.
\begin{itemize}[leftmargin=*]
    \item LLM становится активным агентом, запрашивающим данные профиля ученика через стандартизированный протокол.
    \item Это дает глубокую персонализацию: модель видит именно те пробелы, которые нужно закрыть в текущей теме, без лишнего шума.
\end{itemize}

\subsubsection{Методическая уникальность: Динамические лонгриды}
\noindent Фокус на генерации связных учебных текстов, а не диалогов.
\begin{itemize}[leftmargin=*]
    \item Конспект собирается «на лету»: для проблемных тем теория расширяется и упрощается, для усвоенных — дается в формате краткого обзора (recap).
    \item Ученик получает персональную главу учебника по его потребностям.
\end{itemize}

\subsubsection{Интеграция с ЕГЭ}
\noindent Проект — инструмент для экзамена.
\begin{itemize}[leftmargin=*]
    \item Генерация ограничена Кодификатором ЕГЭ через системные промпты и валидацию MCP.
    \item Это снижает риск галлюцинаций и гарантирует, что время тратится только на проверяемый материал.
\end{itemize}

\newpage

\renewcommand{\bibname}{Список литературы}
\bibliographystyle{plainnat}  % Используем plainnat, записи отформатированы по ГОСТ
\bibliography{references}

\end{document}

